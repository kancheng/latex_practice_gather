
\documentclass[journal]{Imperial_lab_report}

\ifCLASSINFOpdf
\else
\fi

\hyphenation{op-tical net-works semi-conduc-tor}

\begin{document}
\title{Template for Imperial College Physics 1st Year Laboratory Reports}
\author{An I.C. Student}% <-this % stops a space

% The paper headers
\markboth{A. Student}%
{Shell \MakeLowercase{\textit{et al.}}:}

\maketitle
\begin{abstract}
The abstract goes here.
\end{abstract}

\section{Introduction}

\IEEEPARstart{T}{his} demo file is intended to serve as a ``starter file''
for preparing  reports for Imperial College Physics 1st year lab,   using  \LaTeX\ and the  Imperial\_lab\_report.cls class file.
Happy \TeX ing!

\subsection{Subsection Heading Here}
Subsection text here.

% needed in second column of first page if using \IEEEpubid
%\IEEEpubidadjcol

\subsubsection{Subsubsection Heading Here}
Subsubsection text here.

\section{Section 1}
LaTeX is a software system for document preparation. When writing, the writer uses plain text as opposed to the formatted text found in "What You See Is What You Get" word processors like Microsoft Word, LibreOffice Writer and Apple Pages. The writer uses markup tagging conventions to define the general structure of a document (such as article, book, and letter), to stylise text throughout a document (such as bold and italics), and to add citations and cross-references. A TeX distribution such as TeX Live or MiKTeX is used to produce an output file (such as PDF or DVI) suitable for printing or digital distribution.

LaTeX is widely used in academia for the communication and publication of scientific documents in many fields, including mathematics, statistics, computer science, engineering, physics, economics, linguistics, quantitative psychology, philosophy, and political science. It also has a prominent role in the preparation and publication of books and articles that contain complex multilingual materials, such as Sanskrit and Greek. LaTeX uses the TeX typesetting program for formatting its output, and is itself written in the TeX macro language.
\subsubsection{Subsubsection Heading Here}
$$ E = mc^2$$

\section{Section 2}
LaTeX is a software system for document preparation. When writing, the writer uses plain text as opposed to the formatted text found in "What You See Is What You Get" word processors like Microsoft Word, LibreOffice Writer and Apple Pages. The writer uses markup tagging conventions to define the general structure of a document (such as article, book, and letter), to stylise text throughout a document (such as bold and italics), and to add citations and cross-references. A TeX distribution such as TeX Live or MiKTeX is used to produce an output file (such as PDF or DVI) suitable for printing or digital distribution.

LaTeX is widely used in academia for the communication and publication of scientific documents in many fields, including mathematics, statistics, computer science, engineering, physics, economics, linguistics, quantitative psychology, philosophy, and political science. It also has a prominent role in the preparation and publication of books and articles that contain complex multilingual materials, such as Sanskrit and Greek. LaTeX uses the TeX typesetting program for formatting its output, and is itself written in the TeX macro language.

\subsubsection{Subsubsection Heading Here}
$$ E = mc^2$$
\subsubsection{Subsubsection Heading Here}
$$ E = mc^2$$

\section{Section 3}
LaTeX is a software system for document preparation. When writing, the writer uses plain text as opposed to the formatted text found in "What You See Is What You Get" word processors like Microsoft Word, LibreOffice Writer and Apple Pages. The writer uses markup tagging conventions to define the general structure of a document (such as article, book, and letter), to stylise text throughout a document (such as bold and italics), and to add citations and cross-references. A TeX distribution such as TeX Live or MiKTeX is used to produce an output file (such as PDF or DVI) suitable for printing or digital distribution.

LaTeX is widely used in academia for the communication and publication of scientific documents in many fields, including mathematics, statistics, computer science, engineering, physics, economics, linguistics, quantitative psychology, philosophy, and political science. It also has a prominent role in the preparation and publication of books and articles that contain complex multilingual materials, such as Sanskrit and Greek. LaTeX uses the TeX typesetting program for formatting its output, and is itself written in the TeX macro language.
\subsubsection{Subsubsection Heading Here}
$$ E = mc^2$$
\subsubsection{Subsubsection Heading Here}
$$ E = mc^2$$

\section{Section 4}
LaTeX is a software system for document preparation. When writing, the writer uses plain text as opposed to the formatted text found in "What You See Is What You Get" word processors like Microsoft Word, LibreOffice Writer and Apple Pages. The writer uses markup tagging conventions to define the general structure of a document (such as article, book, and letter), to stylise text throughout a document (such as bold and italics), and to add citations and cross-references. A TeX distribution such as TeX Live or MiKTeX is used to produce an output file (such as PDF or DVI) suitable for printing or digital distribution.

LaTeX is widely used in academia for the communication and publication of scientific documents in many fields, including mathematics, statistics, computer science, engineering, physics, economics, linguistics, quantitative psychology, philosophy, and political science. It also has a prominent role in the preparation and publication of books and articles that contain complex multilingual materials, such as Sanskrit and Greek. LaTeX uses the TeX typesetting program for formatting its output, and is itself written in the TeX macro language.
\subsubsection{Subsubsection Heading Here}
$$ E = mc^2$$
\subsubsection{Subsubsection Heading Here}
$$ E = mc^2$$

\section{Conclusion}
The conclusion goes here.

\appendices
\section{Proof of  Einstein's Famous  Equation}
The famous equation $$ E = mc^2$$
can be derived.
% you can choose not to have a title for an appendix
% if you want by leaving the argument blank
\section{}
[Appendix two text goes here.]

% use section* for acknowledgment
\section*{Acknowledgment}

[acknowledgments go here]

% Can use something like this to put references on a page
% by themselves when using endfloat and the captionsoff option.
\ifCLASSOPTIONcaptionsoff
  \newpage
\fi

\bibliographystyle{ieeetran} 

% that's all folks
\end{document}